%<*Figure:multilabel-struc>
\begin{figure}[h]
	\centering
    \includegraphics[width=\textwidth]{figures/multilabel-struc.pdf}
    \caption{Hierarchical structure of the CheXpert dataset. 
	The labels are grouped into 14 categories. Adapted from \citep{Irvin2019}.}
    \label{fig:multilabel-struc}
\end{figure}
%</Figure:multilabel-struc>

%<*Figure:pathologies>
\begin{figure}[htbp]
	\centering
	\begin{subfigure}[t]{0.3\textwidth}
		\includegraphics[width=\textwidth, height=\textwidth]{figures/pathologies/cardiomegaly.png}
		\caption{Cardiomegaly}
		%\label{fig:first}
	\end{subfigure}
	\hfill
	\begin{subfigure}[t]{0.3\textwidth}
		\includegraphics[width=\textwidth, height=\textwidth]{figures/pathologies/pneumonia.jpg}
		\caption{Pneumonia}
		%\label{fig:second}
	\end{subfigure}
	\hfill
	\begin{subfigure}[t]{0.3\textwidth}
		\includegraphics[width=\textwidth, height=\textwidth]{figures/pathologies/rib-fracture.PNG}
		\caption{Rib Fracture}
		%\label{fig:third}
	\end{subfigure}
	\hfill
	\vskip 1pt
	\begin{subfigure}[t]{0.3\textwidth}
		\includegraphics[width=\textwidth, height=\textwidth]{figures/pathologies/pneumothorax.png}
		\caption{Pneumothorax}
		%\label{fig:fourth}
	\end{subfigure}
	\hfill
	\begin{subfigure}[t]{0.3\textwidth}
		\includegraphics[width=\textwidth, height=\textwidth]{figures/pathologies/lung opacity.png}
		\caption{Lung Opacity}
		%\label{fig:fifth}
	\end{subfigure}
	\hfill
	\begin{subfigure}[t]{0.3\textwidth}
		\includegraphics[width=\textwidth, height=\textwidth]{figures/pathologies/pleural effusion.png}
		\caption{Pleural Effusion}
		%\label{fig:sixth}
	\end{subfigure}	
	\caption{Pathologies}
	\label{fig:pathologies}
\end{figure}
%</Figure:pathologies>

%<*Figure:vit-arch>
\begin{figure}[ht]
	\centering
    \includegraphics[width=\textwidth]{figures/vit_arch.drawio.pdf}
	\caption{Architecture of DINO head with ViT Transformer as backbone. Adapted from \citep{Dosovitskiy2020}. 
    The Chest X-ray image is taken from the CheXpert dataset \citep{Irvin2019}.}
	\label{fig:vit-arch}
\end{figure}
%</Figure:vit-arch>

%<*Figure:setting1-chexnorm-v-imgnorm>
\begin{figure}[ht]
	\centering
    \includegraphics[width=\textwidth]{figures/setting1-chexnorm-v-imgnorm.pdf}
	\caption{Comparison between CheXpert and Imagenet normalization for the first setting with SimCLR.
	For the ImageNet normalization, all images were scaled with $\mu=(0.485, 0.456, 0.406)$ and $\sigma=(0.229, 0.224, 0.225)$. 
    The CheXpert normalization stands for a normalization with $\mu=(0.5330, 0.5330, 0.5330)$ and $\sigma=(0.0349, 0.0349, 0.0349)$ \citep[\href{https://github.com/christophbrgr/ood_detection_framework}{\color{Periwinkle}{implementation}}]{Berger2021}.
	Solid lines represent the Imagenet normalization while dashed and dotted lines refer to the CheXpert normalization.
	The ID class is No Finding and OOD classes are the ones that are mentioned in the legend.}
	\label{fig:setting1-chexnorm-v-imgnorm}
\end{figure}
%</Figure:setting1-chexnorm-v-imgnorm>

%<*Figure:labels-used-AUC-setting1>
\begin{figure}
	\centering
    \includegraphics[width=\textwidth]{figures/labels-used-AUC-setting1.pdf}
    \caption{}
    \label{fig:labels-used-AUC-setting1}
\end{figure}
%</Figure:labels-used-AUC-setting1>

%<*Figure:labels-used-AUC-setting2>
\begin{figure}
	\centering
    \includegraphics[width=\textwidth]{figures/labels-used-AUC-setting2.pdf}
	\caption{}
	\label{fig:labels-used-AUC-setting2}
\end{figure}
%</Figure:labels-used-AUC-setting2>

%<*Figure:heatmap-geom>
\begin{figure}
	\centering
	\begin{subfigure}[t]{0.45\textwidth}
		\includegraphics[width=\textwidth, height=0.8\textwidth]{figures/heatmap-geom-acc.pdf}
		\caption{k-nn accuracy}
		%\label{fig:third}
	\end{subfigure}
	\begin{subfigure}[t]{0.45\textwidth}
		\includegraphics[width=\textwidth, height=0.8\textwidth]{figures/heatmap-geom-auroc.pdf}
		\caption{AUROC OOD}
		%\label{fig:third}
	\end{subfigure}
    % \includegraphics[width=\textwidth]{figures/heatmap-geom-acc.pdf}
    \caption{k-nn classification accuracy and AUROC for different geometrical augmentations.}
    \label{fig:heatmap-geom}
\end{figure}
%</Figure:heatmap-geom>

%<*Figure:heatmap-spatial>
\begin{figure}
	\centering
	\begin{subfigure}[t]{0.45\textwidth}
		\includegraphics[width=\textwidth, height=0.8\textwidth]{figures/heatmap-spatial-acc.pdf}
		\caption{k-nn accuracy}
		%\label{fig:third}
	\end{subfigure}
	\begin{subfigure}[t]{0.45\textwidth}
		\includegraphics[width=\textwidth, height=0.8\textwidth]{figures/heatmap-spatial-auroc.pdf}
		\caption{AUROC OOD}
		%\label{fig:third}
	\end{subfigure}
    % \includegraphics[width=\textwidth]{figures/heatmap-spatial-acc.pdf}
    \caption{k-nn classification accuracy and AUROC for different spatial augmentations.}
    \label{fig:heatmap-spatial}
\end{figure}
%</Figure:heatmap-spatial>

%<*Figure:heatmap-ood-auc>
\begin{figure}
	\centering
    \includegraphics[width=\textwidth]{figures/heatmap-ood-auc.pdf}
    \caption{Averaged over both settings. 
	For a detailed breakdown of both settings see the appendix [\textcolor{red}{add REF}].}
    \label{fig:heatmap-ood-auc}
\end{figure}
%</Figure:heatmap-ood-auc>



