%<*Figure:multilabel-struc>
\begin{figure}[h]
	\centering
    \includegraphics[width=\textwidth]{figures/multilabel-struc.pdf}
    \caption{Hierarchical structure of the CheXpert dataset. 
	The labels are grouped into 14 categories.
	The dotted arrow from the Support Devices class indicates, that patients with a support device can also be labeled as No Finding, representing the only source of overlap between the No Finding class and other classes.
	Adapted from \citep{Irvin2019}.}
    \label{fig:multilabel-struc}
\end{figure}
%</Figure:multilabel-struc>

%<*Figure:pathologies>
\begin{figure}[htbp]
	\centering
	\begin{subfigure}[t]{0.3\textwidth}
		\includegraphics[width=\textwidth, height=\textwidth]{figures/pathologies/cardiomegaly.png}
		\caption{Cardiomegaly}
		%\label{fig:first}
	\end{subfigure}
	\hfill
	\begin{subfigure}[t]{0.3\textwidth}
		\includegraphics[width=\textwidth, height=\textwidth]{figures/pathologies/pneumonia.jpg}
		\caption{Pneumonia}
		%\label{fig:second}
	\end{subfigure}
	\hfill
	\begin{subfigure}[t]{0.3\textwidth}
		\includegraphics[width=\textwidth, height=\textwidth]{figures/pathologies/rib-fracture.PNG}
		\caption{Rib Fracture}
		%\label{fig:third}
	\end{subfigure}
	\hfill
	\vskip 1pt
	\begin{subfigure}[t]{0.3\textwidth}
		\includegraphics[width=\textwidth, height=\textwidth]{figures/pathologies/pneumothorax.png}
		\caption{Pneumothorax}
		%\label{fig:fourth}
	\end{subfigure}
	\hfill
	\begin{subfigure}[t]{0.3\textwidth}
		\includegraphics[width=\textwidth, height=\textwidth]{figures/pathologies/lung opacity.png}
		\caption{Lung Opacity}
		%\label{fig:fifth}
	\end{subfigure}
	\hfill
	\begin{subfigure}[t]{0.3\textwidth}
		\includegraphics[width=\textwidth, height=\textwidth]{figures/pathologies/pleural effusion.png}
		\caption{Pleural Effusion}
		%\label{fig:sixth}
	\end{subfigure}	
	\caption{Pathologies}
	\label{fig:pathologies}
\end{figure}
%</Figure:pathologies>

%<*Figure:vit-arch>
\begin{figure}[ht]
	\centering
    \includegraphics[width=\textwidth]{figures/vit_arch.drawio.pdf}
	\caption{Architecture of DINO head with ViT Transformer as backbone. Adapted from \citep{Dosovitskiy2020}. 
    The Chest X-ray image is taken from the CheXpert dataset \citep{Irvin2019}.}
	\label{fig:vit-arch}
\end{figure}
%</Figure:vit-arch>

%<*Figure:simclr-arch>
\begin{figure}[ht]
	\centering
    \includegraphics[width=\textwidth]{figures/simclr-arch.pdf}
	\caption{SimCLR architecture, adapted from \citep{Chen2020}. 
	The Chest X-ray image is taken from the CheXpert dataset \citep{Irvin2019}.
	}
	\label{fig:simclr-arch}
\end{figure}
%</Figure:simclr-arch>

%<*Figure:setting1-chexnorm-v-imgnorm>
\begin{figure}[ht]
	\begin{subtable}[c]{\textwidth}
		\centering
		\begin{tabular}{l c c c c} 
		\toprule
		& \multicolumn{2}{c}{SimCLR} & \multicolumn{2}{c}{DINO}\\
		\cmidrule(lr){2-3} \cmidrule(lr){4-5}
		normalization& CheXpert&ImageNet&CheXpert&ImageNet\\
		\midrule
		Cardiomegaly&57.81&55.26&58.24&\textbf{60.33}\\
		Fracture&53.43&54.53&\textbf{56.43}&55.69\\
		Lung Opacity&55.55&55.49&59.10&\textbf{60.30}\\
		Pleural Effusion&58.48&58.05&\textbf{64.33}&61.56\\
		Pneumothorax&60.10&60.54&57.51&\textbf{61.23}\\
		Support Devices&53.58&53.50&56.97&\textbf{57.44}\\
		\bottomrule
		\end{tabular}
		\caption{AUROC 1 values for last epoch}
		\label{fig:setting1-chexnorm-v-imgnorm-dino-last-epoch}
	\end{subtable}
	\\[.2cm]
	\begin{subfigure}[t]{\textwidth}
		\centering
		\includegraphics{figures/setting1-imagenet-norm-dino-v-simclr-barplot-new.pdf}
		\caption{AUROC 1 values for last epoch for ImageNet normalization}
		\label{fig:setting1-chexnorm-v-imgnorm-dino-barplot}
	\end{subfigure}
    \caption{Comparison between Chexpert and ImageNet normalization for the first setting.
	Table \ref{fig:setting1-chexnorm-v-imgnorm-dino-last-epoch} shows the results for SimCLR and DINO respectively across both normalizations.
	Maximum AUROC values are marked in bold.
	Figure \ref{fig:setting1-chexnorm-v-imgnorm-dino-barplot} depicts the results for both models restricted to ImageNet normalization only.
	The dotted line refers to the performance equivalent to the chance level at 50\% AUROC.
	For the ImageNet normalization, all images were scaled with $\mu=(0.485, 0.456, 0.406)$ and $\sigma=(0.229, 0.224, 0.225)$. 
    The CheXpert normalization stands for a normalization with $\mu=(0.5330, 0.5330, 0.5330)$ and $\sigma=(0.0349, 0.0349, 0.0349)$ \citep[\href{https://github.com/christophbrgr/ood_detection_framework}{\color{Periwinkle}{implementation}}]{Berger2021}.
	The ID class is No Finding and OOD classes are the ones that are mentioned in the plot.}
\end{figure}
%</Figure:setting1-chexnorm-v-imgnorm>

%<*Figure:auroc-vs-k-and-temp>
\begin{figure}[ht]
	\begin{subfigure}[t]{\textwidth}
		\centering
		\includegraphics{figures/auroc-vs-k.pdf}
		\caption{}
		%AUROC in relation to the number of nearest neighbours used for feature similarity.
		\label{fig:auroc-vs-k}
	\end{subfigure}
	\begin{subfigure}[t]{\textwidth}
		\centering
		\includegraphics{figures/auroc-vs-temp.pdf}
		\caption{}
		%AUROC of settings 1-3 compared to the temperature.
		\label{fig:auroc-vs-temp}
	\end{subfigure}
	\caption{The values for setting 1 were calculated as a top-3 average of the six OOD pathologies.
	$N_i$ refers to the length of the in-distribution train dataset of the i-th setting, where $i \in \{1,2,3\}$.
	The temperature is set to 1 for all values in figure \ref{fig:auroc-vs-k}.}
\end{figure}
%</Figure:auroc-vs-k-and-temp>

%<*Figure:dino-rotate-augs-v-default-augs>
\begin{figure}[ht]
	\centering
	\includegraphics{figures/dino-rotate-augs-vs-default-augs-new.pdf}
\end{figure}
%</Figure:dino-rotate-augs-v-default-augs>


%<*Figure:labels-used-AUC-setting1>
\begin{figure}
	\centering
    \includegraphics[width=\textwidth]{figures/labels-used-AUC-setting1.pdf}
    \caption{}
    \label{fig:labels-used-AUC-setting1}
\end{figure}
%</Figure:labels-used-AUC-setting1>

%<*Figure:labels-used-AUC-setting2>
\begin{figure}
	\centering
    \includegraphics[width=\textwidth]{figures/labels-used-AUC-setting2.pdf}
	\caption{}
	\label{fig:labels-used-AUC-setting2}
\end{figure}
%</Figure:labels-used-AUC-setting2>

%<*Figure:heatmap-geom>
\begin{figure}
	\centering
	\begin{subfigure}[t]{0.45\textwidth}
		\includegraphics[width=\textwidth, height=0.8\textwidth]{figures/heatmap-geom-acc.pdf}
		\caption{k-nn accuracy}
		%\label{fig:third}
	\end{subfigure}
	\begin{subfigure}[t]{0.45\textwidth}
		\includegraphics[width=\textwidth, height=0.8\textwidth]{figures/heatmap-geom-auroc.pdf}
		\caption{AUROC OOD}
		%\label{fig:third}
	\end{subfigure}
    % \includegraphics[width=\textwidth]{figures/heatmap-geom-acc.pdf}
    \caption{k-nn classification accuracy and AUROC for different geometrical augmentations.}
    \label{fig:heatmap-geom}
\end{figure}
%</Figure:heatmap-geom>

%<*Figure:heatmap-spatial>
\begin{figure}
	\centering
	\begin{subfigure}[t]{0.45\textwidth}
		\includegraphics[width=\textwidth, height=0.8\textwidth]{figures/heatmap-spatial-acc.pdf}
		\caption{k-nn accuracy}
		%\label{fig:third}
	\end{subfigure}
	\begin{subfigure}[t]{0.45\textwidth}
		\includegraphics[width=\textwidth, height=0.8\textwidth]{figures/heatmap-spatial-auroc.pdf}
		\caption{AUROC OOD}
		%\label{fig:third}
	\end{subfigure}
    % \includegraphics[width=\textwidth]{figures/heatmap-spatial-acc.pdf}
    \caption{k-nn classification accuracy and AUROC for different spatial augmentations.}
    \label{fig:heatmap-spatial}
\end{figure}
%</Figure:heatmap-spatial>

%<*Figure:heatmap-ood-auc>
\begin{figure}
	\centering
    \includegraphics[width=\textwidth]{figures/heatmap-ood-auc.pdf}
    \caption{Averaged over both settings. 
	For a detailed breakdown of both settings see the appendix [\textcolor{red}{add REF}].}
    \label{fig:heatmap-ood-auc}
\end{figure}
%</Figure:heatmap-ood-auc>



