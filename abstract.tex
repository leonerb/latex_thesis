%%% Die folgende Zeile nicht ändern!
\section*{\ifthenelse{\equal{\sprache}{deutsch}}{Zusammenfassung}{Abstract}}
%%% Zusammenfassung:
Out-of-distribution detection (OoDD) is a strategy that aims to detect unsuitable input samples that could lead to model failure at prediction time.
This failure should be avoided in medical applications, where reliable predictions are crucial.
A recent work has used supervised classifiers that are trained on the in-distribution (ID) to extract confidence scores and to identify out-of-distribution (OOD) samples \citep{Berger2021}.
In this thesis, self-supervised learning (SSL) methods are applied to compute representations of chest-X-rays without presumed labels.
A nearest-neighbour based feature similarity between training- and test features \citep{Michels2023,Sun2022} is used to extract scores for OoDD from the representations of SimCLR \citep{Chen2020} and DINO \citep{Caron2021}.
Three different dataset splits with different ID and OOD classes are considered.  
It is shown that SSL methods can perform unsupervised OoDD on chest X-rays with only minor modifications to the used augmentations. 
The applied methods are less performant than supervised methods and the detection accuracy varies depending on the included pathologies.
Two strategies improved the performance of SSL methods for OoDD.
One approach is fine-tuning with a supervised classifier, that can lead to comparable performance to supervised methods but also uses the same labels.
In the second case, expanding the training data with labelled OOD samples \citep{Fort2021,Hendrycks2018} can also enhance the performance, even without fine-grained access to OOD class labels.
In summary, this thesis evaluates SSL methods for OoDD on chest radiographs.
SSL methods initially have lower performance, assuming label knowledge can bridge this gap and improve the identification of OOD samples. 
This highlights the importance of labelled samples to achieve higher accuracy in OoDD for chest X-ray images and motivates future research to reduce the need for labels.