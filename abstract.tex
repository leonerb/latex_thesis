%%% Die folgende Zeile nicht ändern!
\section*{\ifthenelse{\equal{\sprache}{deutsch}}{Zusammenfassung}{Abstract}}
% introduction
Out-of-distribution detection is a strategy that aims to detect input samples that could lead to model failure at prediction time.
Model failure must be avoided in medical applications where reliable predictions are critical.
% methods
%\\
A recent work \citep{Berger2021} has used score-based methods to identify out-of-distribution (OOD) samples for chest X-ray images.
Score-based OOD detection relies on the prediction of a supervised classifier trained on the in-distribution (ID) classes \citep{Yang2021}.
In this thesis, the self-supervised learning (SSL) paradigms SimCLR \citep{Chen2020} and DINO \citep{Caron2021} are used to compute representations of chest X-rays without presumed labels.
The architecture of both methods combines a feature extractor and a specific projection head.
Also, a Vision Transformer (ViT) \citep{Dosovitskiy2020} is used as a feature extractor and pretrained ViTs are applied to compute training features and test features from chest X-rays.
Then, the nearest neighbor feature similarity between both sets of features is used as a score for OOD detection \citep{Michels2023,Sun2022}.
% exp setup
%\\
Three different dataset splits of CheXpert \citep{Irvin2019} are considered.
The first one includes healthy patients as ID and patients with one out of six pathologies as OOD.
The second (ID: \textit{Cardiomegaly}, \textit{Pneumothorax}; OOD: \textit{Fracture}) and third setting (ID: \textit{Lung Opacity}, \textit{Pleural Effusion}; OOD: \textit{Fracture}, \textit{Pneumonia}) are taken from \citep{Berger2021}.
Pretraining was performed on all settings and on CheXpert.
To compare with supervised baseline methods, pretrained ViTs were also fine-tuned on ID classes.
% results
%\\
The main findings of this thesis are:
(i) Expanding the training data with only a few unlabeled OOD samples can improve the performance of pretrained ViTs for OOD detection on setting one.
(ii) The performance of pretrained ViTs fine-tuned with ID data is comparable to the supervised baseline performance in \citep{Berger2021} on settings two and three.
(iii) Adding rotations to the image augmentations improves the ID accuracy between \textit{Cardiomegaly} and \textit{Pneumothorax}, but not the feature similarity-based OOD detection accuracy against \textit{Fracture} samples.
% conclusion
%\\
Overall, SSL methods can be used to pretrain ViTs for OOD detection on chest X-rays but are less performant than supervised baseline methods if not fine-tuned on ID data.
The reliance on labeled samples for OOD detection on chest X-rays motivates further research to reduce the need for labels.
