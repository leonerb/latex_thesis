%\setcounter{secnumdepth}{4} %Nummerieren bis in die 4. Ebene
%\setcounter{tocdepth}{4} %Inhaltsverzeichnis bis zur 4. Ebene

\pagestyle{headings}

\sloppy % LaTeX ist dann nicht so streng mit der Silbentrennung
%~ \MakeShortVerb{\§}

\parindent0mm
\parskip0.5em


{
\textwidth170mm
\oddsidemargin30mm
\evensidemargin30mm
\addtolength{\oddsidemargin}{-1in}
\addtolength{\evensidemargin}{-1in}

\parskip0pt plus2pt

% Die Raender muessen eventuell fuer jeden Drucker individuell eingestellt
% werden. Dazu sind die Werte fuer die Abstaende `\oben' und `\links' zu
% aendern, die von mir auf jeweils 0mm eingestellt wurden.

%\newlength{\links} \setlength{\links}{10mm}  % hier abzuaendern
%\addtolength{\oddsidemargin}{\links}
%\addtolength{\evensidemargin}{\links}

\begin{titlepage}
\vspace*{-1.5cm}
\raisebox{17mm}{
    \begin{minipage}[t]{70mm}
        \begin{center}
            %\selectlanguage{german}
            {\Large INSTITUT FÜR\\INFORMATIK\\}
            \vspace{3mm}
            {\small Universitätsstr. 1 \hspace{5ex} D--40225 Düsseldorf\\}
        \end{center}
    \end{minipage}
}
\hfill
\raisebox{7mm}{
    \includegraphics[width=130pt]{logos/HHU_Logo}}
\vspace{14em}

% Titel
\begin{center}
    \baselineskip=55pt
    \textbf{\huge \titel}
    \baselineskip=0 pt
\end{center}

%\vspace{7em}

\vfill

% Autor
\begin{center}
    \textbf{\Large
        \bearbeiter
    }
\end{center}

\vspace{35mm}

% Prüfungsordnungs-Angaben
\begin{center}
%\selectlanguage{german}

%%%%%%%%%%%%%%%%%%%%%%%%%%%%%%%%%%%%%%%%%%%%%%%%%%%%%%%%%%%%%%%%%%%%%%%%%
% Ja, richtig, hier kann die BA-Vorlage zur MA-Vorlage gemacht werden...
% (nicht mehr nötig!)
%%%%%%%%%%%%%%%%%%%%%%%%%%%%%%%%%%%%%%%%%%%%%%%%%%%%%%%%%%%%%%%%%%%%%%%%%
{\Large \arbeit}

\vspace{2em}
\ifthenelse{\equal{\sprache}{deutsch}}{
    \begin{tabular}[t]{ll}
    Beginn der Arbeit:& \beginndatum \\
    Abgabe der Arbeit:& \abgabedatum \\
    Gutachter:         & \erstgutachter \\
    & \zweitgutachter \\
}{
    \begin{tabular}[t]{ll}
    Date of issue:& \beginndatum \\
    Date of submission:& \abgabedatum \\
    Reviewers:         & \erstgutachter \\
    & \zweitgutachter \\
}
\end{tabular}
\end{center}

\end{titlepage}

}

%%%%%%%%%%%%%%%%%%%%%%%%%%%%%%%%%%%%%%%%%%%%%%%%%%%%%%%%%%%%%%%%%%%%%
\clearpage
%\begin{titlepage}
%    ~                % eine leere Seite hinter dem Deckblatt
%\end{titlepage}
%%%%%%%%%%%%%%%%%%%%%%%%%%%%%%%%%%%%%%%%%%%%%%%%%%%%%%%%%%%%%%%%%%%%%
\clearpage
\begin{titlepage}
    \vspace*{\fill}

    \section*{Erklärung}

    %%%%%%%%%%%%%%%%%%%%%%%%%%%%%%%%%%%%%%%%%%%%%%%%%%%%%%%%%%%
    % Und hier ebenfalls ggf. BA durch MA ersetzen...
    % (Auch nicht mehr nötig!)
    %%%%%%%%%%%%%%%%%%%%%%%%%%%%%%%%%%%%%%%%%%%%%%%%%%%%%%%%%%%

    Hiermit versichere ich, dass ich diese \arbeit{}
    selbstständig verfasst habe. Ich habe dazu keine anderen als die
    angegebenen Quellen und Hilfsmittel verwendet.

    \vspace{25 mm}

    \begin{tabular}{lc}
        Düsseldorf, den \abgabedatum \hspace*{2cm} & \underline{\hspace{6cm}} \\
                                                   & \bearbeiter
    \end{tabular}

    \vspace*{\fill}
\end{titlepage}

%%%%%%%%%%%%%%%%%%%%%%%%%%%%%%%%%%%%%%%%%%%%%%%%%%%%%%%%%%%%%%%%%%%%%
% Leerseite bei zweiseitigem Druck
%%%%%%%%%%%%%%%%%%%%%%%%%%%%%%%%%%%%%%%%%%%%%%%%%%%%%%%%%%%%%%%%%%%%%

%\ifthenelse{\equal{\zweiseitig}{twoside}}{\clearpage\begin{titlepage}
%        ~\end{titlepage}}{}

%%%%%%%%%%%%%%%%%%%%%%%%%%%%%%%%%%%%%%%%%%%%%%%%%%%%%%%%%%%%%%%%%%%%%
\clearpage
\begin{titlepage}

    %%% Die folgende Zeile nicht ändern!
\section*{\ifthenelse{\equal{\sprache}{deutsch}}{Zusammenfassung}{Abstract}}
% introduction
Out-of-distribution detection is a strategy that aims to detect input samples that could lead to model failure at prediction time.
Model failure must be avoided in medical applications where reliable predictions are critical.
% methods
%\\
A recent work \citep{Berger2021} has used score-based methods to identify out-of-distribution (OOD) samples for chest X-ray images.
Score-based OOD detection relies on the prediction of a supervised classifier trained on the in-distribution (ID) classes \citep{Yang2021}.
In this thesis, the self-supervised learning (SSL) paradigms SimCLR \citep{Chen2020} and DINO \citep{Caron2021} are used to compute representations of chest X-rays without presumed labels.
The architecture of both methods combines a feature extractor and a specific projection head.
Also, a Vision Transformer (ViT) \citep{Dosovitskiy2020} is used as a feature extractor and pretrained ViTs are applied to compute training features and test features from chest X-rays.
Then, the nearest neighbor feature similarity between both sets of features is used as a score for OOD detection \citep{Michels2023,Sun2022}.
% exp setup
%\\
Three different dataset splits of CheXpert \citep{Irvin2019} are considered.
The first one includes healthy patients as ID and patients with one out of six pathologies as OOD.
The second (ID: \textit{Cardiomegaly}, \textit{Pneumothorax}; OOD: \textit{Fracture}) and third setting (ID: \textit{Lung Opacity}, \textit{Pleural Effusion}; OOD: \textit{Fracture}, \textit{Pneumonia}) are taken from \citep{Berger2021}.
Pretraining was performed on all settings and on CheXpert.
To compare with supervised baseline methods, pretrained ViTs were also fine-tuned on ID classes.
% results
%\\
The main findings of this thesis are:
(i) Expanding the training data with only a few unlabeled OOD samples can improve the performance of pretrained ViTs for OOD detection on setting one.
(ii) The performance of pretrained ViTs fine-tuned with ID data is comparable to the supervised baseline performance in \citep{Berger2021} on settings two and three.
(iii) Adding rotations to the image augmentations improves the ID accuracy between \textit{Cardiomegaly} and \textit{Pneumothorax}, but not the feature similarity-based OOD detection accuracy against \textit{Fracture} samples.
% conclusion
%\\
Overall, SSL methods can be used to pretrain ViTs for OOD detection on chest X-rays but are less performant than supervised baseline methods if not fine-tuned on ID data.
The reliance on labeled samples for OOD detection on chest X-rays motivates further research to reduce the need for labels.



    %%%%%%%%%%%%%%%%%%%%%%%%%%%%%%%%%%%%%%%%%%%%%%%%
    % Untere Titelmakros. Editieren Sie diese Datei nur, wenn Sie sich
    % ABSOLUT sicher sind, was Sie da tun!!!
    %%%%%%%%%%%%%%%%%%%%%%%%%%%%%%%%%%%%%%%%%%%%%%%
    \vspace*{\fill}
\end{titlepage}

%%%%%%%%%%%%%%%%%%%%%%%%%%%%%%%%%%%%%%%%%%%%%%%%%%%%%%%%%%%%%%%%%%%%%
% Leerseite bei zweiseitigem Druck
%%%%%%%%%%%%%%%%%%%%%%%%%%%%%%%%%%%%%%%%%%%%%%%%%%%%%%%%%%%%%%%%%%%%%
%\ifthenelse{\equal{\zweiseitig}{twoside}}
%{\clearpage\begin{titlepage}~\end{titlepage}}{}
%%%%%%%%%%%%%%%%%%%%%%%%%%%%%%%%%%%%%%%%%%%%%%%%%%%%%%%%%%%%%%%%%%%%%
%\clearpage 
\setcounter{page}{1}
\pagenumbering{roman}
\setcounter{tocdepth}{2}
\tableofcontents

%\enlargethispage{\baselineskip}
%\clearpage
%%%%%%%%%%%%%%%%%%%%%%%%%%%%%%%%%%%%%%%%%%%%%%%%%%%%%%%%%%%%%%%%%%%%%
% Leere Seite, falls Inhaltsverzeichnis mit ungerader Seitenzahl und
% doppelseitiger Druck
%%%%%%%%%%%%%%%%%%%%%%%%%%%%%%%%%%%%%%%%%%%%%%%%%%%%%%%%%%%%%%%%%%%%%
%\ifthenelse{ \( \equal{\zweiseitig}{twoside} \and \not \isodd{\value{page}} \)}
%{\pagebreak \thispagestyle{empty} \cleardoublepage}{\clearpage}


% Kapitel soll bei doppelseitigem Druck immer auf der rechten (ungeraden) Seite anfangen (thx @ Philipp Grawe)
% https://tex.stackexchange.com/a/223387
\ifthenelse{\boolean{\sectionforcestartright}}
{\let\oldsection\section % Store \section in \oldsection
    \renewcommand{\section}{\cleardoublepage\oldsection}}
{}