%%%%%%%%%%%%%%%%%%%%%%%%%%%%%%%%%%%%%%%%%%%%%%%%%%%%%%%%%%%%%%%%%%%%%%%%
% Uni Duesseldorf
% Lehrstuhl fuer Datenbanken und Informationssysteme
% Vorlage fuer Bachelor-/Masterarbeiten
% Optimiert fuer den Original-Latex-Kompiler LATEX.EXE (LaTeX=>PS=>PDF)
%%%%%%%%%%%%%%%%%%%%%%%%%%%%%%%%%%%%%%%%%%%%%%%%%%%%%%%%%%%%%%%%%%%%%%%%
% Ueberarbeitung für pdflatex (LaTeX=>PDF)
%%%%%%%%%%%%%%%%%%%%%%%%%%%%%%%%%%%%%%%%%%%%%%%%%%%%%%%%%%%%%%%%%%%%%%%%
% Vorlage Changelog:
% 10.09.2015 (Matthias Liebeck): Nummerierung des Inhaltsverzeichnis nun römisch, Beispiel für einen Anhang eingebaut, \raggedbottom hinter sections eingefügt
% 11.07.2018 (Matthias Liebeck): Ersetzung des Bibliographiestils, Einsatz von Biber
% 04.09.2018 (Matthias Liebeck):
%   * Bibtex: unnötige Bibtexfelder beim Rendern ausblenden (thx @ Markus Brenneis)
%   * ngerman: "et al." im BibTeX für drei oder mehr Autoren
%   * Neuer Befehl \sectionforcestartright: Sections immer rechts beginnen (thx @ Philipp Grawe)
%   * ngerman: Deutsche Anführungszeichen im Literaturverzeichnis (thx @ Markus Brenneis)
%   * ngerman: Deutsche Anführungszeichen im Literaturverzeichnis (thx @ Markus Brenneis)
% 16.10.2018 (Matthias Liebeck): Zwei fixes an \sectionforcestartright (thx @ Markus Brenneis)
%%%%%%%%%%%%%%%%%%%%%%%%%%%%%%%%%%%%%%%%%%%%%%%%%%%%%%%%%%%%%%%%%%%%%%%%
%%%% BEGINN EINSTELLUNG FUER DIE ARBEIT. UNBEDINGT ERFORDERLICH! %%%%%%%
%%%%%%%%%%%%%%%%%%%%%%%%%%%%%%%%%%%%%%%%%%%%%%%%%%%%%%%%%%%%%%%%%%%%%%%%
% Geben Sie Ihren Namen hier an:

\newcommand{\bearbeiter}{Leon Erbrich}

% Geben Sie hier den Titel Ihrer Arbeit an:
\newcommand{\titel}{Tbd}

% Geben Sie das Datum des Beginns und Ende der Bachelorarbeit ein:
\newcommand{\beginndatum}{15. Oktober 2022}
\newcommand{\abgabedatum}{28.~Februar~2023}

% Geben Sie die Namen des Erst- und Zweitgutachters an:
\newcommand{\erstgutachter}{Prof. Dr.~Markus Kollmann}
\newcommand{\zweitgutachter}{tbd}

% Falls Sie die Arbeit zweiseitig ausdrucken wollen,
% benutzen Sie die folgende Zeile mit
% \AN fuer zweiseitigen Druck
% \AUS fuer einseitigen Druck
\newcommand{\zweiseitig}{\AN}
% true fuer biber, false fuer klassischen Zitierstil
%\newcommand{\biber}{false}
\newcommand{\biber}{true}

% Falls Sections immer rechts beginnen sollen. Gerade für Masterarbeiten
% interessant. Bei kurzen Bachelorarbeiten eher weniger zu verwenden.
%\newcommand{\sectionforcestartright}{false}
\newcommand{\sectionforcestartright}{true}

% Falls die Arbeit in englischer Sprache verfasst
% werden soll, dann benutzen Sie die folgende Zeile mit
% englisch fuer englische Sprache
% deutsch fuer deutsche Sprache
\newcommand{\sprache}{englisch}

% Hier wird eingestellt, ob es sich bei der Arbeit um eine Bachelor-
% oder Masterarbeit handelt (unpassendes auskommentieren!):
%\newcommand{\arbeit}{Bachelorarbeit}
\newcommand{\arbeit}{Masterarbeit}


%%%%%%%%%%%%%%%%%%%%%%%%%%%%%%%%%%%%%%%%%%%%%%%%%%%%%%%%%%%%%%%%%%%%%%%%
%%%% ENDE EINSTELLUNGEN %%%%%%%%%%%%%%%%%%%%%%%%%%%%%%%%%%%%%%%%%%%%%%%%
%%%%%%%%%%%%%%%%%%%%%%%%%%%%%%%%%%%%%%%%%%%%%%%%%%%%%%%%%%%%%%%%%%%%%%%%

% Die folgende Zeile NICHT EDITIEREN oder loeschen


%%%%%%%%%%%%%%%%%%%%%%%%%%%%%%%%%%%%%%%%%%%%%%%%%%%%%%%%%%%
% Obere Titelmakros. Editieren Sie diese Datei nur, wenn
% Sie sich ABSOLUT sicher sind, was Sie da tun!!!
% (Z.B. zum Abaendern der BA-Vorlage in eine MA-Vorlage)
% Uni Duesseldorf
% Lehrstuhl fuer Datenbanken und Informationssysteme
% Version 2.2 - 2.3.2010
%%%%%%%%%%%%%%%%%%%%%%%%%%%%%%%%%%%%%%%%%%%%%%%%%%%%%%%%%%%
\newcommand{\AN}{twoside}
\newcommand{\AUS}{}


%\newcommand{\englisch}{}
%\newcommand{\deutsch}{\usepackage[german]{babel}}

%% Die folgenden auskommentierten Optionen dienen der automatischen
%% Erkennung des Latex-Kompilers und dem Setzen der davon abhängigen
%% Einstellungen. Bei Problem z.B. mit dem Einbinden von verschiedenen
%% Grafiktypen bei Verwendung von PdfLatex oder Latex, einfach die
%% verschiedenen \usepackage(s) ausprobieren. (Mit diesen Einstellungen
%% funktionierte diese Vorlage bei der Verwenundg von latex.exe als
%% Kompiler bei den meisten Studierenden.)

%\newif\ifpdf \ifx\pdfoutput\undefined
%\pdffalse % we are not running pdflatex
%\else
%\pdfoutput=1 % we are running pdflatex
%\pdfcompresslevel=9 % compression level for text and image;
%\pdftrue \fi

\documentclass[11pt, a4paper, \zweiseitig]{article}
% width of document: 360.0pt.
\usepackage{ifthen}


%\usepackage[iso]{umlaute}
\usepackage[utf8]{inputenc}
\usepackage{palatino} % palatino Schriftart
%\usepackage{makeidx} % um ein Index zu erstellen
\usepackage[nottoc]{tocbibind}
\usepackage[T1]{fontenc} %fuer richtige Trennung bei Umlauten
\usepackage{fancybox} % fuer die Rahmen
\usepackage{shortvrb}
\usepackage{url}
\usepackage[usenames,dvipsnames]{xcolor}
\usepackage{amsmath}
\usepackage[colorlinks,citecolor=Periwinkle,linkcolor=black, hyperindex]{hyperref} %anklickbares Inhaltsverzeichnis
\usepackage{setspace}
\usepackage{booktabs}
\usepackage{adjustbox}
\usepackage{subcaption}
\usepackage{pifont}
\usepackage{multirow}
\usepackage{pgf}
\usepackage{array}
\newcommand{\cmark}{\ding{51}}%
\newcommand{\xmark}{\ding{55}}%
\usepackage{catchfilebetweentags}
\newcommand{\loadTable}[1]{ % define command to load figures
   \ExecuteMetaData[tables.tex]{#1} % call the package macro to load chunk from file
}
\newcommand{\loadFigure}[1]{ % define command to load figures
   \ExecuteMetaData[figures.tex]{#1} % call the package macro to load chunk from file
}
% following is a fix for the catchfilebetweentags package: otherwise a new line swallows the whitespace
\usepackage{etoolbox}
\makeatletter
\patchcmd{\CatchFBT@Fin@l}{\endlinechar\m@ne}{}
  {}{\typeout{Unsuccessful patch!}}
\makeatother
\newcommand{\citestyle}{nature}
\ifthenelse{\boolean{\biber}}{
  % only needed for biber
  \usepackage[style=\citestyle,natbib=true,backend=biber,mincitenames=1,maxcitenames=2,maxbibnames=99,uniquelist=false]{biblatex}

  % https://tex.stackexchange.com/a/334703/8850
  \AtEveryBibitem{%
    \clearfield{issn}
    \clearfield{isbn}
    \clearfield{doi}
    \clearfield{location}
    \clearlist{location}
    \clearlist{address}

    \ifentrytype{online}{}{% Remove url except for @online
      \clearfield{url}
    }
  }
}
{}%no else
% Falls es bei \citet ein Komma zwischen Name und Jahr gibt:
% https://tex.stackexchange.com/questions/312539/unwanted-comma-between-author-and-year-using-citet-command
% (thx @ Markus Brenneis)
%\DeclareDelimFormat[cbx@textcite]{nameyeardelim}{\addspace}



\ifthenelse{\equal{\sprache}{deutsch}}{
  \usepackage[ngerman]{babel}
  % Bibtex u.a -> et al.
  \ifthenelse{\boolean{\biber}}{
    \DefineBibliographyStrings{ngerman}{
      andothers = {{et\,al\adddot}},
    }
    \newcommand{\references}{Literatur}
  }
  {} % do nothing when not using biber
  \usepackage[autostyle, german=quotes]{csquotes} % Deutsche Anführungszeichen im Literaturverzeichnis (thx @ Markus Brenneis)

}{ \newcommand{\references}{References}}

\usepackage{a4wide} % ganze A4 Weite verwenden



%\ifpdf
%\usepackage[pdftex,xdvi]{graphicx}
%\usepackage{thumbpdf} %thumbs fuer Pdf
%\usepackage[pdfstartview=FitV]{hyperref} %anklickbares Inhaltsverzeichnis
%\else
%\usepackage[dvips,xdvi]{graphicx}
\usepackage{graphicx}

%\fi

\newcommand{\redt}[1] {
  \textcolor{red}{#1}}

\newcommand{\oranget}[1] {
  \textcolor{orange}{#1}}

\newcommand{\purplet}[1] {
  \textcolor{purple}{#1}}

%%%%%%%%%%%%%%%%%%%%%%% Massangaben fuer die Arbeit %%%%%%%%%%%%%%%
\setlength{\textwidth}{15cm}

\setlength{\oddsidemargin}{35mm}
\setlength{\evensidemargin}{25mm}

\addtolength{\oddsidemargin}{-1in}
\addtolength{\evensidemargin}{-1in}

\RequireBibliographyStyle{\citestyle}

\ifthenelse{\boolean{\biber}}{\addbibresource{references.bib}}{}

%\makeindex
\begin{document}
%\setcounter{secnumdepth}{2} %Nummerieren bis in die 4. Ebene
%\setcounter{tocdepth}{1} %Inhaltsverzeichnis bis zur 2. Ebene

\pagestyle{headings}

\sloppy % LaTeX ist dann nicht so streng mit der Silbentrennung
%~ \MakeShortVerb{\§}

\parindent0mm
\parskip0.5em


{
\textwidth170mm
\oddsidemargin30mm
\evensidemargin30mm
\addtolength{\oddsidemargin}{-1in}
\addtolength{\evensidemargin}{-1in}

\parskip0pt plus2pt

% Die Raender muessen eventuell fuer jeden Drucker individuell eingestellt
% werden. Dazu sind die Werte fuer die Abstaende `\oben' und `\links' zu
% aendern, die von mir auf jeweils 0mm eingestellt wurden.

%\newlength{\links} \setlength{\links}{10mm}  % hier abzuaendern
%\addtolength{\oddsidemargin}{\links}
%\addtolength{\evensidemargin}{\links}

\begin{titlepage}
\vspace*{-1.5cm}
\raisebox{17mm}{
    \begin{minipage}[t]{70mm}
        \begin{center}
            %\selectlanguage{german}
            {\Large INSTITUT FÜR\\INFORMATIK\\}
            \vspace{3mm}
            {\small Universitätsstr. 1 \hspace{5ex} D--40225 Düsseldorf\\}
        \end{center}
    \end{minipage}
}
\hfill
\raisebox{7mm}{
    \includegraphics[width=130pt]{logos/HHU_Logo}}
\vspace{14em}

% Titel
\begin{center}
    \baselineskip=55pt
    \textbf{\huge \titel}
    \baselineskip=0 pt
\end{center}

%\vspace{7em}

\vfill

% Autor
\begin{center}
    \textbf{\Large
        \bearbeiter
    }
\end{center}

\vspace{35mm}

% Prüfungsordnungs-Angaben
\begin{center}
%\selectlanguage{german}

%%%%%%%%%%%%%%%%%%%%%%%%%%%%%%%%%%%%%%%%%%%%%%%%%%%%%%%%%%%%%%%%%%%%%%%%%
% Ja, richtig, hier kann die BA-Vorlage zur MA-Vorlage gemacht werden...
% (nicht mehr nötig!)
%%%%%%%%%%%%%%%%%%%%%%%%%%%%%%%%%%%%%%%%%%%%%%%%%%%%%%%%%%%%%%%%%%%%%%%%%
{\Large \arbeit}

\vspace{2em}
\ifthenelse{\equal{\sprache}{deutsch}}{
    \begin{tabular}[t]{ll}
    Beginn der Arbeit:& \beginndatum \\
    Abgabe der Arbeit:& \abgabedatum \\
    Gutachter:         & \erstgutachter \\
    & \zweitgutachter \\
}{
    \begin{tabular}[t]{ll}
    Date of issue:& \beginndatum \\
    Date of submission:& \abgabedatum \\
    Reviewers:         & \erstgutachter \\
    & \zweitgutachter \\
}
\end{tabular}
\end{center}

\end{titlepage}

}

%%%%%%%%%%%%%%%%%%%%%%%%%%%%%%%%%%%%%%%%%%%%%%%%%%%%%%%%%%%%%%%%%%%%%
\clearpage
%\begin{titlepage}
%    ~                % eine leere Seite hinter dem Deckblatt
%\end{titlepage}
%%%%%%%%%%%%%%%%%%%%%%%%%%%%%%%%%%%%%%%%%%%%%%%%%%%%%%%%%%%%%%%%%%%%%
\clearpage
\begin{titlepage}
    \vspace*{\fill}

    \section*{Erklärung}

    %%%%%%%%%%%%%%%%%%%%%%%%%%%%%%%%%%%%%%%%%%%%%%%%%%%%%%%%%%%
    % Und hier ebenfalls ggf. BA durch MA ersetzen...
    % (Auch nicht mehr nötig!)
    %%%%%%%%%%%%%%%%%%%%%%%%%%%%%%%%%%%%%%%%%%%%%%%%%%%%%%%%%%%

    Hiermit versichere ich, dass ich diese \arbeit{}
    selbstständig verfasst habe. Ich habe dazu keine anderen als die
    angegebenen Quellen und Hilfsmittel verwendet.

    \vspace{25 mm}

    \begin{tabular}{lc}
        Düsseldorf, den \abgabedatum \hspace*{2cm} & \underline{\hspace{6cm}} \\
                                                   & \bearbeiter
    \end{tabular}

    \vspace*{\fill}
\end{titlepage}

%%%%%%%%%%%%%%%%%%%%%%%%%%%%%%%%%%%%%%%%%%%%%%%%%%%%%%%%%%%%%%%%%%%%%
% Leerseite bei zweiseitigem Druck
%%%%%%%%%%%%%%%%%%%%%%%%%%%%%%%%%%%%%%%%%%%%%%%%%%%%%%%%%%%%%%%%%%%%%

%\ifthenelse{\equal{\zweiseitig}{twoside}}{\clearpage\begin{titlepage}
%        ~\end{titlepage}}{}

%%%%%%%%%%%%%%%%%%%%%%%%%%%%%%%%%%%%%%%%%%%%%%%%%%%%%%%%%%%%%%%%%%%%%
\clearpage
\begin{titlepage}

    %%% Die folgende Zeile nicht ändern!
\section*{\ifthenelse{\equal{\sprache}{deutsch}}{Zusammenfassung}{Abstract}}
% introduction
Out-of-distribution detection is a strategy that aims to detect input samples that could lead to model failure at prediction time.
Model failure must be avoided in medical applications where reliable predictions are critical.
% methods
%\\
A recent work \citep{Berger2021} has used score-based methods to identify out-of-distribution (OOD) samples for chest X-ray images.
Score-based OOD detection relies on the prediction of a supervised classifier trained on the in-distribution (ID) classes \citep{Yang2021}.
In this thesis, the self-supervised learning (SSL) paradigms SimCLR \citep{Chen2020} and DINO \citep{Caron2021} are used to compute representations of chest X-rays without presumed labels.
The architecture of both methods combines a feature extractor and a specific projection head.
Also, a Vision Transformer (ViT) \citep{Dosovitskiy2020} is used as a feature extractor and pretrained ViTs are applied to compute training features and test features from chest X-rays.
Then, the nearest neighbor feature similarity between both sets of features is used as a score for OOD detection \citep{Michels2023,Sun2022}.
% exp setup
%\\
Three different dataset splits of CheXpert \citep{Irvin2019} are considered.
The first one includes healthy patients as ID and patients with one out of six pathologies as OOD.
The second (ID: \textit{Cardiomegaly}, \textit{Pneumothorax}; OOD: \textit{Fracture}) and third setting (ID: \textit{Lung Opacity}, \textit{Pleural Effusion}; OOD: \textit{Fracture}, \textit{Pneumonia}) are taken from \citep{Berger2021}.
Pretraining was performed on all settings and on CheXpert.
To compare with supervised baseline methods, pretrained ViTs were also fine-tuned on ID classes.
% results
%\\
The main findings of this thesis are:
(i) Expanding the training data with only a few unlabeled OOD samples can improve the performance of pretrained ViTs for OOD detection on setting one.
(ii) The performance of pretrained ViTs fine-tuned with ID data is comparable to the supervised baseline performance in \citep{Berger2021} on settings two and three.
(iii) Adding rotations to the image augmentations improves the ID accuracy between \textit{Cardiomegaly} and \textit{Pneumothorax}, but not the feature similarity-based OOD detection accuracy against \textit{Fracture} samples.
% conclusion
%\\
Overall, SSL methods can be used to pretrain ViTs for OOD detection on chest X-rays but are less performant than supervised baseline methods if not fine-tuned on ID data.
The reliance on labeled samples for OOD detection on chest X-rays motivates further research to reduce the need for labels.



    %%%%%%%%%%%%%%%%%%%%%%%%%%%%%%%%%%%%%%%%%%%%%%%%
    % Untere Titelmakros. Editieren Sie diese Datei nur, wenn Sie sich
    % ABSOLUT sicher sind, was Sie da tun!!!
    %%%%%%%%%%%%%%%%%%%%%%%%%%%%%%%%%%%%%%%%%%%%%%%
    \vspace*{\fill}
\end{titlepage}

%%%%%%%%%%%%%%%%%%%%%%%%%%%%%%%%%%%%%%%%%%%%%%%%%%%%%%%%%%%%%%%%%%%%%
% Leerseite bei zweiseitigem Druck
%%%%%%%%%%%%%%%%%%%%%%%%%%%%%%%%%%%%%%%%%%%%%%%%%%%%%%%%%%%%%%%%%%%%%
%\ifthenelse{\equal{\zweiseitig}{twoside}}
%{\clearpage\begin{titlepage}~\end{titlepage}}{}
%%%%%%%%%%%%%%%%%%%%%%%%%%%%%%%%%%%%%%%%%%%%%%%%%%%%%%%%%%%%%%%%%%%%%
%\clearpage 
\setcounter{page}{1}
\pagenumbering{roman}
\setcounter{tocdepth}{2}
\setcounter{secnumdepth}{4}
\tableofcontents

%\enlargethispage{\baselineskip}
%\clearpage
%%%%%%%%%%%%%%%%%%%%%%%%%%%%%%%%%%%%%%%%%%%%%%%%%%%%%%%%%%%%%%%%%%%%%
% Leere Seite, falls Inhaltsverzeichnis mit ungerader Seitenzahl und
% doppelseitiger Druck
%%%%%%%%%%%%%%%%%%%%%%%%%%%%%%%%%%%%%%%%%%%%%%%%%%%%%%%%%%%%%%%%%%%%%
%\ifthenelse{ \( \equal{\zweiseitig}{twoside} \and \not \isodd{\value{page}} \)}
%{\pagebreak \thispagestyle{empty} \cleardoublepage}{\clearpage}


% Kapitel soll bei doppelseitigem Druck immer auf der rechten (ungeraden) Seite anfangen (thx @ Philipp Grawe)
% https://tex.stackexchange.com/a/223387
\ifthenelse{\boolean{\sectionforcestartright}}
{\let\oldsection\section % Store \section in \oldsection
    \renewcommand{\section}{\cleardoublepage\oldsection}}
{}
\pagenumbering{arabic}
\setcounter{page}{1}

%%%%%%%%%%%%%%%%%%%%%%%%%%%%%%%%%%%%%%%%%%%%%%%%%%%%%%%%%%%%%%%%%%%%%%%%
%%%% BEGINN TEXTTEIL %%%%%%%%%%%%%%%%%%%%%%%%%%%%%%%%%%%%%%%%%%%%%%%%%%%
%%%%%%%%%%%%%%%%%%%%%%%%%%%%%%%%%%%%%%%%%%%%%%%%%%%%%%%%%%%%%%%%%%%%%%%%

%%%%%%%%%%%%%%%%%%%%%%%%%%%%%%%%%%%%%%%%%%%%%%%%%%%%%%%%%%%%%%%%%%%%%%%%
% Text entweder direkt hier hinein schreiben oder, im Sinne der
% besseren Uebersichtlich- und Bearbeitbarkeit mittels \input die
% einzelnen Textteile hier einbinden.
%%%%%%%%%%%%%%%%%%%%%%%%%%%%%%%%%%%%%%%%%%%%%%%%%%%%%%%%%%%%%%%%%%%%%%%%

\section{Introduction}
\raggedbottom
Most Machine Learning (ML) applications operate under closed-world assumptions.
Models are trained on the \textit{train set}, while the true performance is evaluated on a held-out \textit{test set}, which was not seen in the training process.
It is implicitly assumed, that the \textit{test set} serves as a proxy for the true performance which would be encountered, when models are deployed in real-world applications.
\par
Recent works question the generalization performance of ML models trained under closed-world assumptions \textcolor{red}{[SOURCES]}.
The worse performance is often a result from unsuitable input data.
In general the data which is known to the classifier (train and validation set) is described as \textit{in-distribution} (ID) while new, often unknown data represents samples from \textit{out-of-distribution} (OOD).
The task of out-of-distribution detection (OoDD) is to find inputs deviating from the training distribution.
Therefore, a reliable model output cannot be guaranteed, and a separate consideration may be reasonable.
\par
These detection tasks are especially important in high-risk-environments such as medicine, where human intervention is often required.
X-rays are a relevant imaging technique in medicine and frequently used as a diagnostic tool.
\par
According to \citep{Cao2020} one can generally distinguish three different use-cases.
In the first use-case, images that are unrelated to x-ray-images should be filtered out, e.g. they might originate from a different imaging procedure.
In the second use-case the OOD images are x-ray-images, but they are acquired incorrectly and the image characteristics might be flawed (e.g. high contrast, rotation etc.).
The third use-case considers x-ray images of unseen or novel pathologies as OOD samples.
In this thesis, the focus is mainly on the third use case, which is unsolved to a large extent for x-ray-images.
As a collection of chest-x-ray images, the CheXpert dataset is used \citep{Irvin2019} and the ID-OOD splits are adapted from \citep{Berger2021}.
\par
For example, the OOD detection of novel diseases only scores close to a random guess of 50\% AUROC across a large sample of unsupervised and supervised methods \citep{Cao2020}.
However, the authors of \citep{Berger2021} achieve better performance using a comparable but different data set, proving that increases above chance level are technically possible.
But, these improvements are based on supervised classifiers leading to limitations in the real world, because the acquisition of labels often has high costs. 
\par
Self-supervised learning (SSL) does not assume prior label knowledge and learning objectives such as DINO \citep{Caron2021} or SimCLR \citep{Chen2020} show impressive results in the field of computer vision [SOURCES].
Vision Transformers (ViTs) are used as the backbone architecture of the SSL methods.
Features of the input images are extracted and performance is typically evaluated in downstream tasks.
\par
The performance of SSL methods on out-of-distribution detection for chest-x-ray images is examined and compared to supervised methods.
If applicable, what are the necessary changes to apply SSL methods to chest-x-ray images?
It can be a limitation in the clinical context, if for example healthy control patients should represent the in-distribution class and patients with pathologies are considered as OOD samples.
In this case, supervised classifier assume the presence of at least two in-distribution classes.
A setting that simulates this scenario is developed. 
A second class can be for example constructed by simulating a pathology through image distortions.
Another approach could be to reveal a certain percentage of OOD samples that would then represent at least one additional class (outlier exposure).
Both options are evaluated.

%%%%%%%%%%%%%%%%%%%%%%%%%%%%%%%%%%%%%%%%%%%%%%%%%%%%%%%%%%%%%%%%%%%%%%%%
%%%% ENDE TEXTTEIL %%%%%%%%%%%%%%%%%%%%%%%%%%%%%%%%%%%%%%%%%%%%%%%%%%%%%
%%%%%%%%%%%%%%%%%%%%%%%%%%%%%%%%%%%%%%%%%%%%%%%%%%%%%%%%%%%%%%%%%%%%%%%%

\clearpage

% Entfernen Sie das Kommentar aus der nachfolgenden Zeile, falls Sie einen Anhang in der Arbeit verwenden wollen. 
% Beachten Sie, dass Sie sich im Verlauf der Arbeit mit \ref{...} (z.B. \ref{anhang:zusatz1}) auf den Anhang beziehen.
%\newpage
\appendix
\section{Anhang}

\subsection*{Zusatzteil 1} \label{anhang:zusatz1}

This is an appendix.

\clearpage

\ifthenelse{\boolean{\biber}}{ %with biber do
	\DeclareNameAlias{sortname}{given-family}
	\printbibliography[heading=bibintoc, title=\references]
}{ %without biber do
	\bibliography{references}
	\bibliographystyle{alphadin}
}
%\vspace*{\fill}

\clearpage

\listoffigures

\listoftables

%\pagebreak

%\printindex
\end{document}
