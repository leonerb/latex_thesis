\section{Discussion}
\begin{itemize}
    \item summarize findings
    \item interpretation and implication
        \begin{itemize}
            \item setting 1: 
                \begin{itemize}
                    \item custom augs unsuccessful
                    \item unsupervised outlier exposure however worked even with quite small label fractions $\rightarrow$ which is useful in the real world as although access to OOD samples is needed for better results, no fine-grained OOD class labels are needed
                    \item fracture class still no large improvements $\rightarrow$ fracture class may look quite similar to control class and ROI is too small?, try out other normalizations for fracture [SOURCES].
                \end{itemize} 
        \item setting 2: 
            \begin{itemize}
                \item SSL learning achieves comparable results when finetuned, so number of labeled samples can be reduced, SOTA performance of ODIN could not be reached
            \end{itemize}
        \item setting 3: 
                \begin{itemize}
                    \item no satisfactory results could be achieved
                    \item possible reasons: similarity between the super ID class lung opacity and OOD class pneumonia is high
                    \item better approach: learn the sub labels of the super ID class for better separability $\rightarrow$ still requires label knowledge
                    \item success of applying outlier exposure to setting 1 may transfer to setting 3 and label knowledge could be reduced $\rightarrow$ further studies
                \end{itemize}
        \end{itemize}
    \item limitations
        \begin{itemize}
            \item limited scope: findings should be evaluated on different Chest X-ray datasets e.g. ChestX-ray8 \citep{Wang2017} and across the ID-OOD splits developed in \citep{Cao2020} to ensure validity and generalizability
            \item SupCon: did not result in satisfactory results $\rightarrow$ reason: did not find the right hyperparameter setup, training with AdamW optimizer probably more successful
            \item resolution is low $\rightarrow$ work on methods that can integrate images of higher resolution [SOURCES suchen]
            \item did not leverage radiology reports $\rightarrow$ multimodal methods e.g. CLIP [QUELLE] can combine text and image data and it was recently shown, that they are powerful OOD detectors [pre-print NIKOLAS]. Already applied to x-rays [QUELLE].
        \end{itemize} 
    \item further research directions:
        \begin{itemize}
            \item pretrain on larger datasets + also different imaging procedures e.g. CT scans
            \item use DINO v2 
            \item explore how the global crops scale influences the performance
            \item how does the ViT patch size relate to the performance
        \end{itemize}
    \item conclusion:
        \begin{itemize}
            \item SSL can be adopted to novel disease ood detection for chest-xray images with few adjustments needed
            \item if finetuned, the results are comparable to supervised baseline methods
            \item interestingly, for outlier exposure, no access to fine-grained OOD class labels is needed, different to standard Computer-vision benchmarks [SOURCE]
            \item Overall this represents a promising research direction for future work.
        \end{itemize}
\end{itemize}